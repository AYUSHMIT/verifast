\documentclass{article}

\title{Assignment:\\
Verification of an Internet Game Server}

\begin{document}

\maketitle

\section{Assignment}
The goal of this assignment is to verify a game server. More specifically, you must add annotations to \texttt{GameServer.c} such that the VeriFast program verifier accepts the program. Modifying the source code itself or using \texttt{assume} statements is not permitted. It is not necessary for specifications to be complete; it suffices to prove the absence of memory errors and data races.

Your final solution must be submitted to \texttt{jan.smans@cs.kuleuven.be} by March 7th 2010. It suffices to send the annotated version of \texttt{GameServer.c}. If you get stuck, you can request hints by sending an email describing your problem to \texttt{jan.smans@cs.kuleuven.be}. In fact, we strongly encourage asking for help if you get stuck: it's better to ask questions and submit a more complete assignment, than to give up!

After submitting your solution, we will shedule an appointment. In this appointment, you must defend your solution and demonstrate your understanding of VeriFast.

\section{Extra}
It is not necessary to complete the optional (but harder!) items in the list below to succeed for this assignment. However, completing them will be rewarded.

\begin{itemize}
  \item Verification of the function \texttt{create\_game\_last} is optional. If you do not verify \texttt{create\_game\_last}, you must use the contract \texttt{requires false; ensures true;} and place the call to this function in comments.
  \item Extend \texttt{GameServer.c} to provide additional menu options, different games, etc. Verify your extensions. If you choose to do so, submit both the original \texttt{GameServer.c} and your \texttt{GameServer-extended.c}. The reward for extensions will be determined by their complexity and how hard their correctness proofs are.
\end{itemize}

\section{Hints}

\begin{itemize}
  \item The use of fractional permissions is required to complete this assignment. Consumption of an assertion \texttt{[?f]P(x, y, z)} consumes the first chunk \texttt{[g]P(x, y, z)} in the heap and binds \texttt{f} to {g}. A heap chunk \texttt{[g]P(x, y, z)} can be split into two heap chunks using the \texttt{split\_fraction} ghost command: \texttt{//@ split\_fraction P(x, y, z)}.
\end{itemize}

\section{Compiling and Running the Server}

\subsection{Windows}

\subsubsection*{Compiling}
To compile the server on Windows, the Microsoft C/C++ compiler (\texttt{cl.exe}) must be installed. This compiler comes with visual studio. Enter the following command on the command-line:\newline\newline
\texttt{cl.exe /D "WIN32" *.c ws2\_32.lib}\newline\newline
The program can also be compiled in Visual Studio by opening GameServer.sln. Visual C++ Express is available for free on the Microsoft website.
\subsubsection*{Running}
To run the server, run the executable generated by the compiler. Clients can connect via telnet\footnote{Telnet is not installed by default on Windows Vista and Windows 7. Go to \texttt{Control Panel->Programs and Features->Turn Windows Features on or off to install telnet.}} on port 1234:
\newline\newline
\texttt{telnet localhost 1234}

\subsection{Linux}

\subsubsection*{Compiling}
To compile the server on Linux, the GNU C compiler (\texttt{gcc}) must be installed. Enter the following command on the command-line:\newline\newline
\texttt{gcc -pthread *.c}
\subsubsection*{Running}
To run the server, run the executable generated by the compiler. Clients can connect via telnet on port 1234:\newline\newline
\texttt{telnet localhost 1234}

\end{document}
